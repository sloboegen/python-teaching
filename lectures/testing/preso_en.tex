\documentclass[compress]{beamer}

\usepackage{minted}
\usepackage[american,russian]{babel}

\usetheme{Warsaw}

\usecolortheme[named=black]{structure}
\usefonttheme{structuresmallcapsserif}

\title{Testing in Python}
\author{Evgeniy Slobodkin}
\institute{ITMO Univeristy}
\date{Fall 2023}

\begin{document}


\frame{\titlepage}


\begin{frame}
\frametitle{Disclaimer}

\begin{itemize}
    \item<1-> Testing is challenging
    \item<2-> Mainly we will focus on unit tests
    \item<3-> Many facts in this lecture are language-agnostic
\end{itemize}

\end{frame}


\begin{frame}
\frametitle{Why do we need tests?}

\begin{itemize}
  \item<1-> Verify that new code works as expected
  \item<2-> Verify that old code works as expected (\textit{Regression testing})
  \item<3-> Document your code
  \item<4-> Help newcomers with diving into the project
  \item<5-> Discipline us (!)
\end{itemize}

\end{frame}


\begin{frame}[fragile]
\frametitle{doctest module\footnotemark[1]}
\footnotetext[1]{\url{https://docs.python.org/3/library/doctest.html}}

\begin{minted}[fontsize=\small]{Python}
# main.py
def second_max(xs: list[int]) -> int | None:
    """Finds a second max in a given list if it exists.

    >>> second_max([])
    None
    >>> second_max([1, 1])
    None
    >>> second_max([3, 1, 2])
    2
    """
    xs = list(set(xs))
    return xs[-2] if len(xs) >= 2 else None
\end{minted}

\vspace{0.3in}

Possible way to run your doctests: \texttt{python -m doctest main.py}

\end{frame}


\begin{frame}[fragile]
\frametitle{When to use doctest?}

\begin{itemize}
    \item Test your documentation
    \item Do not test logic by \texttt{doctest}
\end{itemize}

\end{frame}


\begin{frame}[fragile]
\frametitle{unittest module}

\begin{itemize}
  \item Testing Framework in Python standard library
  \item Inspired by JUnit
  \item TODO
\end{itemize}

\end{frame}


\begin{frame}[fragile]
\frametitle{pytest}

\begin{columns}[T]
\begin{column}{.8\textwidth}
    \begin{itemize}
    \item Initially was originated from the PyPy project
    \item Many powerful features
    \item One \textbf{assert} for any case
    \item There is only reason to use \texttt{unittest} instead of \texttt{pytest}: no third-party libraries are allowed in your project
    \end{itemize}
\end{column}
\begin{column}{.2\textwidth}
    \includegraphics[width=\textwidth]{pytest_logo.png}
\end{column}
\end{columns}
  
\end{frame}

\begin{frame}[fragile]
\frametitle{Test parametrization\footnotemark[1]}
\footnotetext[1]{https://docs.pytest.org/en/latest/example/parametrize.html}

\begin{overprint}
\onslide<1>
\begin{center}
\begin{minted}{Python}
def test_second_max() -> None:
    """Test ensures that second_max works correctly."""
    datas = (
        ([], None),
        ([1], None),
        ([1, 1], None),
        ([2, 1], 1),
        (list(range(10)), 8),
    )
    for xs, sm in datas:
        assert second_max(xs) == sm
\end{minted}
\end{center}

\onslide<2>
\begin{center}
\Huge Do not do this!
\end{center}

\onslide<3>
\begin{center}
\begin{minted}{Python}
@pytest.mark.parametrize(
    "xs,sm",
    [
        ([], None),
        ([1], None),
        ([1, 1], None),
        ([2, 1], 1),
        (list(range(10)), 8),
    ],
)
def test_second_max(xs: list[int], sm: int) -> None:
    """Test ensures that second_max works correctly."""
    assert second_max(xs) == sm
\end{minted}
\end{center}
\end{overprint}

\end{frame}

\begin{frame}[fragile]
\frametitle{Fixtures\footnotemark[1]}
\footnotetext[1]{https://docs.pytest.org/en/latest/explanation/fixtures.html}

\begin{overprint}

\onslide<1>
From pytest doc: \textit{In testing, a fixture provides a defined, reliable and consistent context for the tests.}

\vspace{0.3in}

\begin{columns}[T]
\begin{column}{.5\textwidth}
\begin{minted}[fontsize=\scriptsize]{Python}
@pytest.fixture(scope='session')
def course_timetable() -> Timetable:
    return {
        'C++': 'Fri, 10AM',
        'Python': 'Wed, 9AM',
        'Algorithms': 'Mon, 2PM',
        'Algebra': 'Mon, 5PM',
        'ML': 'Tue, 10AM',
    }
\end{minted}
\end{column}
\begin{column}{.5\textwidth}
\begin{minted}[fontsize=\scriptsize]{Python}
@pytest.fixture(scope='function')
def vasya_student() -> Student:
    return Student(
        name='Vasya',
        courses={'C++', 'Python'},
    )
\end{minted}
\end{column}
\end{columns}


\onslide<2>
\begin{center}
\begin{minted}[fontsize=\footnotesize]{Python}
def test_vasya_timetable(
    vasya_student: Student,
    course_timetable: Timetable,
) -> None:
    vt = generate_student_timetable(
        vasya_student,
        course_timtable,
    )

    assert vt == 'Python: Wed, 9AM\nC++: Fri, 10AM'
\end{minted}
\end{center}

\end{overprint}


\end{frame}


\begin{frame}[fragile]
\frametitle{Factory as Fixture Pattern}
% \frametitle{Factory as Fixture Pattern\footnotemark[1]}
% \footnotetext[1]{https://docs.pytest.org/en/6.2.x/fixture.html#factories-as-fixtures}

\begin{overprint}
\onslide<1>
\begin{center}
\begin{minted}[fontsize=\scriptsize]{Python}
StudentFactory: TypeAlias = Callable[[str, list[str]], Student]

@pytest.fixture(scope='session')
def student_factory() -> StudentFactory:
    names = ('Petya', 'Vasya', ..., 'Grisha')
    courses = ('C++', 'Python', ..., 'Algorithms')

    def _factory(name=None, courses=None) -> Student:
        cnt = random.randint(a=3,b=8)
        name = random.choice(names) if name is None else name
        courses = random.sample(courses, cnt) if courses is None else courses
        return Student(name=name, courses=courses)

    return _factory
\end{minted}
\end{center}

\onslide<2>
\begin{center}
\begin{minted}[fontsize=\scriptsize]{Python}
def test_scholarship_value(student_factory: StudentFactory) -> None:
    student = student_factory()
    assert 0 <= calc_scholarship_rub(student) <= 15_000


def test_register_for_course(student_factory: StudentFactory) -> None:
    students = [student_factory() for _ in range(10)]
    cpp_students = [s for s in students if 'C++' in s.courses]

    register_students(students)
    assert len(get_students_for_course('C++')) == len(cpp_students)


def test_empty_courses_is_forbidden(
    student_factory: StudentFactory,
    course_timetable: Timetable,
) -> None:
    student = student_factory(courses=[])
    with pytest.raises(EmptyCoursesError):
        _ = generate_student_timetable(student, course_timetable)
\end{minted}
\end{center}
\end{overprint}

\end{frame}

\begin{frame}[fragile]
\frametitle{Libraries for Fake Data Generation}

The most common are \href{https://mimesis.name/en/master/}{Mimesis\footnotemark[1]} and \href{https://faker.readthedocs.io/en/master/}{Faker\footnotemark[2]}

\footnotetext[1]{\url{https://mimesis.name/en/master/}}
\footnotetext[2]{\url{https://faker.readthedocs.io/en/master/}}

\begin{center}
    \begin{itemize}
        \item Provide generators for different entities
        \item Provide localization
        \item Do not forget about the seed!\footnotemark[3]
    \end{itemize}
\end{center}

\footnotetext[3]{\url{https://mimesis.name/en/master/random_and_seed.html}}

\vspace{0.2in}

\begin{minted}[fontsize=\scriptsize]{Python}
# from mimesis import ...
>>> mf = Field(locale=Locale.RU, seed=random.seed())
>>> mf('person.first_name')
'Родослав'
>>> mf('person.occupation')
'Учитель'
>>> mf('person.email')
'animation1848@yahoo.com'
\end{minted}

\end{frame}

% \begin{frame}[fragile]
% \frametitle{Advanced Factory as Fixture}

% \begin{minted}[fontsize=\scriptsize]{Python}
% class StudentData(TypedDict):
%     name: str
%     email: str
%     gender: Literal['M', 'F']

% @pytest.fixture(scope='session')
% def student_factory() -> StudentFactory:
%     def _factory(**fields: Unpack[StudentData]) -> StudentData:
%         mf = Field(locale=Locale.RU, seed=random.seed())
%         gender = Gender.FEMALE if random.getrandbits(1) else Gender.MALE
%         name = mf('person.first_name', gender=gender)
%         email = mf('person.email')
%         schema = Schema(
%             schema=lambda: {
%                 'name': name,
%                 'gender': 'F' if gender is Gender.FEMALE else 'M',
%                 'email': email,
%             }
%         )

%         return {**(schema.create()[0]), **fields}

%     return _factory
% \end{minted}

% \end{frame}


\begin{frame}[fragile]
\frametitle{Test organization}

\textit{One of the examples of project organization (my preferred)}:

\begin{columns}[T]
\begin{column}{.5\textwidth}
\begin{minted}[fontsize=\small]{Shell}
src
    __init__.py
tests
    conftest.py
    plugins
        student_gen.py
        timetable_gen.py
    test_student
        test_student_data.py
        conftest.py
\end{minted}
\end{column}

\begin{column}{.5\textwidth}
\begin{minted}[fontsize=\footnotesize]{Python}
# tests/conftest.py
pytest_plugins = [
    "plugins.student_gen",
    "plugins.timetable_gen",
]

def pytest_configure(config):
    ...
\end{minted}
\end{column}
\end{columns}

\end{frame}


\begin{frame}[fragile]
\frametitle{Mocking}

\begin{itemize}
    \item How to test code that calls other services or sends messages (for instance, Telegram Bot)?
    \item How to test code that calls some still unimplemented functions (but with known API)?
\end{itemize}

\vspace{0.2in}

\begin{center}
\includegraphics[scale=0.2]{thinking_emoji.png}
\end{center}

\end{frame}


\begin{frame}[fragile]
\frametitle{Mock it!}

\begin{center}
\begin{minted}[fontsize=\scriptsize]{Python}
# publisher.py
class Publisher:
    def __init__(bot: Bot, admin: User, subs: list[User]) -> None:
        self._bot = bot
        self._admin = admin
        self._subs = subs

    def publish(self, announcement: str) -> None:
        for sub in self.subs:
            self._bot.send_message(
                sub.id, f'{sub.name}, hi!\n\n{announcement}'
            )
        self._bot.send_message(self._admin.id, 'All done!')

# test_publisher.py
def test_publish() -> None:
    mock = create_autospec(Bot)
    publisher = Publisher(mock, User(1, 'Vasya'), [User(2, 'Petya')]

    publisher.publish()

    assert mock.send_message.mock_calls[0].args == (2, ANY)
    assert mock.send_message.mock_calls[1].args == (1, ANY)
\end{minted}
\end{center}

\end{frame}


\begin{frame}[fragile]
\frametitle{Different mocks\footnotemark[1]}

\footnotetext[1]{\url{https://docs.python.org/3/library/unittest.mock.html}}

\begin{itemize}
    \item Mock
    \item MagicMock
    \item NonCallableMock
    \item NonCallableMagicMock
    \item PropertyMock
    \item AsyncMock
\end{itemize}
\end{frame}


\begin{frame}[fragile]
\frametitle{Useful pytest plugins}

\begin{itemize}
    \item pytest-randomly\footnote[1]{\url{https://pypi.org/project/pytest-randomly/}}
    \item pytest-cov\footnote[2]{\url{https://pypi.org/project/pytest-cov/}}
    \item pytest-asyncio\footnote[3]{\url{https://pypi.org/project/pytest-asyncio/}}
    \item pytest-xdist\footnote[4]{\url{https://pypi.org/project/pytest-xdist/}}
    \item pytest-mock\footnote[5]{\url{https://pypi.org/project/pytest-mock/}}
    \item pytest-timeout\footnote[6]{\url{https://pypi.org/project/pytest-timeout/}}
\end{itemize}

\end{frame}


\begin{frame}[fragile]
\frametitle{Property-based testing}

\end{frame}

\begin{frame}[fragile]
\frametitle{Hypothesis}

\end{frame}

\begin{frame}[fragile]
\frametitle{Example}

\end{frame}

\end{document}
